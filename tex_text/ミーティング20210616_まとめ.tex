\documentclass[ 10pt]{jsarticle}

\usepackage[dvipdfm]{graphicx}      % include this line if your document contains figures

\usepackage{algorithmicx,algpseudocode}
\usepackage{mathtools}
\usepackage{subfigure}
\usepackage{cases}
\usepackage{bm}
%\usepackage{txfonts}
\usepackage{comment}
%\usepackage{mathtools}
\usepackage{amsmath}
\usepackage{balance}
\usepackage{amssymb}
\usepackage{amsfonts}
\usepackage{comment}
\graphicspath{{./pic/}}
\usepackage{algorithm}
%\usepackage{algorithmic}
%\usepackage[linesnumbered, ruled]{algorithm2e}
%\usepackage{url}
\newcommand{\qedwhite}{\hfill \ensuremath{\Box}}
\newtheorem{myrem}{Remark}
\newtheorem{dfn}{Definition}
\newtheorem{pbm}{Problem}
\newtheorem{exa}{Example}
\newcommand{\rdef}[1]{Definition\,\ref{#1}}
\newcommand{\req}[1]{\eqref{#1}} 
\newcommand{\rpro}[1]{Problem\,\ref{#1}}
\newcommand{\rsec}[1]{Section\,\ref{#1}}
\newcommand{\rfig}[1]{Fig.\,\ref{#1}} 

\newcommand{\argmax}{\mathop{\rm arg~max}\limits}
\newcommand{\argmin}{\mathop{\rm arg~min}\limits}
\newcommand{\Count}{{\sf Count}}
\newcommand{\tick}{{\sf tick}}
\newcommand{\Len}{{\sf L}}
\renewcommand{\algorithmicrequire}{\textbf{Input:}}
\renewcommand{\algorithmicensure}{\textbf{Output:}}

\DeclareMathSymbol{\smallslash}{\mathord}{operators}{32}
\newcommand{\negexcl}{\mathrel{\smallslash\mkern-5mu{!}}}

\setcounter{pbm}{2}
%\setcounter{algorithm}{1}

\title{ミーティング(2021/6/16)のまとめ}
\author{衣川 琢磨}

\begin{document}
\maketitle
2021/6/16のミーティングで決まったことと、それらをもとに考えたことをまとめます。

\section{階層TDESの問題設定の変更}
TACに提出した論文では、Problem 2 の問題設定においてAlgorithm 2を用いて階層TDESの動作計画をtop-down的に上位TDESから求めていった。
%Problem \ref{pbm3} とAlgorithm \ref{alg4}は\rsec{app}に示しています。
しかし、この手法では以下のような課題が指摘された。
\begin{itemize}
\item
上位TDESと下位TDESの整合性がとれていない。
\item
階層TDES全体で何を行っているのかが分かりにくい。
\end{itemize}
そこで、階層TDESの動作計画をbottom-up的に下位TDESから行っていくアルゴリズムに修正する。
二階層の階層TDESを考えるとして、この節では仕様の与え方と問題設定を修正し、\rsec{app}で動作計画を行う方法・アルゴリズムについて説明する。

bottom-up的に動作計画を行うため、上位TDESと下位TDESへの仕様の与え方を表\ref{spec}のように変更する。
%
\begin{table}[htb]
\centering
\caption{仕様の与え方の変更点}
\label{spec}
\begin{tabular}{c||c|c}
 &変更前 & 変更後 \\\hline\hline
Soft制約&全ての階層で与える&最も上位のTDESにのみ与える\\\hline
Modular of a specification&導入した&導入しない\\\hline
パラメータ化される仕様 &下位TDESへの仕様&最も上位のTDESへの仕様
\end{tabular}
\end{table}
%

変更後の仕様の与え方を詳細に記述していく。

一つの上位TDES(pTDES) $G$と$M$個の下位TDES(cTDES) $G_1,\ldots,G_M$があるとする。$i\in\{1,\ldots,M\}$として、$G$の状態$s_i$と$G_i$が対応付けられているとする。
また、上位TDES $G$の単位時間を$\tick_p$、下位TDES $G_i$の単位時間を$\tick_i$で表記し、$\tick_i$が経過したことを表す事象を$\textit{tick}_i$とする。

それぞれの下位TDES $G_i$に与えるHard制約を$\phi_i$と表記する。$t_i$は、$G_i$で$\phi_i$を満たすために必要な時間(事象$\textit{tick}_i$の数)とする。

変更後では、上位TDESに与える仕様をパラメータ化する。
パラメーターを要素としてもつベクトルを$\bm{m}=(m_1,\ldots,m_M)\in\mathbb{Z}_{\geq 0}^M$とする。
$m_i$は$t_i$によって値が決まる。
$\bm{m}$によりパラメータ化されたticked LTL${}_f$式を$\varphi(\bm{m})$と表記する。

上位TDES $G$にはHard制約を表すticked LTL${}_f$式$\phi(\bm{m})$と、$N$個のSoft制約を表すticked LTL${}_f$式$\psi_1(\bm{m}),\ldots,\psi_N(\bm{m})$とその重み$w_j$からなるタプルの集合$\Psi(\bm{m})=\{(\psi_j(\bm{m}),w_j)|\ j\in\{1,\ldots,N\}\}$を与える。

ここまでの変更で、階層TDESの動作計画を求める問題設定を以下のように定める。
\begin{pbm}\label{pbm3}
Given a two-level hierarchical TDES $\mathcal{G}$, a parameterized hard constraint $\phi(\bm{m})$ and a set $\Psi(\bm{m})$ of parameterized soft constraints for $G$, a  hard constraint $\phi_i$ for each $G_i$, where $i \in [1,\ M]$, and the length $\Len(>0)$ for $G$, find finite executions $\pi_i$ of $G_i$, then find finite executions $\pi$ of $G$ with the length $\Len$.
% a function $f_i$
%
\end{pbm}
%
このProblem 3の文章は修正が必要だと考えています。
%
%
\section{新しいアルゴリズムについて}\label{app}
%
まず、以下に示す天井関数$\lceil x \rceil$を導入する。
\begin{align}
\lceil x \rceil =\min\{n\in\mathbb{Z}|\ x\leq n\}.
\end{align}
%

Problem \ref{pbm3}を解くためのアルゴリズムの流れを以下にまとめる。
%
\begin{enumerate}
\item
すべての$i\in\{1,\ldots,M\}$について、
\begin{enumerate}
\item\label{1-a}
下位TDES $G_i$のHard制約$\phi_i$を満たす実行列(execution)$\pi_i$を求める。
\item\label{1-b}
$\pi_i$から$\phi_i$を満たすのにかかった時間を求め、$t_i$とする
\end{enumerate}
\item\label{2}
それぞれの$i$について、$m_i=\left\lceil t_i\frac{\tick_i}{\tick_p} \right\rceil = \left\lceil \frac{t_i}{K_i}\right\rceil $を求める($K_i$は提出した論文の式15で導入)
\item
$G$の仕様に$\bm{m}=(m_1,\ldots,m_M)$の値を代入
\item
$G$の動作計画を行う
\end{enumerate}
%
\ref{1-a}で$\pi_i$を求めるために実行列の長さ$\Len_i$を与える。
最短時間でHard制約を満たすために、$\pi_i$を求める最適化問題の目的関数に
\begin{align}
\mbox{min:}~\sum_{k=1}^{\Len_i}z_e(k)
\end{align}
を追加する。$z_e(k)$は$\pi_i$の$k$番目の事象が$\textit{tick}_i$なら1、それ以外の事象なら0をとる二値変数である(TACに提出した論文の式6から式9で導入しています)。

また、前回のミーティングではSoft制約を下位TDESには与えない、となりましたが、以下のような形式でSoft制約$\Psi_i=\{(\psi_{i,j},w_{i,j})|\ j\in\{1,\ldots,N_i\}\}$を組み込むことは可能だと感じました。
\begin{align}\label{obj1}
\mbox{min:}~\sum_{k=1}^{\Len_i}z_e(k) + \left(-\sum_{j=1}^{N_i} w_{i,j}\cdot z_{\psi_{i,j}}\right)
\end{align}
または、
\begin{align}
\mbox{min:}~\alpha\sum_{k=1}^{\Len_i}z_e(k) + \left\{-(1-\alpha)\sum_{j=1}^{N_i} w_{i,j}\cdot z_{\psi_{i,j}}\right\},~0\leq \alpha\leq 1
\end{align}
%
\req{obj1}では、時間とSoft制約のトレードオフ(Soft制約は満たしたいけど、余計な時間がかかる)について、以下のようなことが考えられる。
\begin{itemize}
\item
$w_{i,j}>1$なら1単位時間($1\tick_i$)を余分にかけて$\psi_{i,j}$を満たす価値がある。
\item
$w_{i,j_1}+w_{i,j_2}>3$なら3単位時間($3\tick_i$)を余分にかけて$\psi_{i,j_1}$と$\psi_{i,j_2}$を満たす価値がある。
\end{itemize}
%

\ref{1-b}を行う時、$t_i$を簡単に求めるために、
\begin{align}
t_i=\mbox{($\pi_i$の$s_0$から$s_{\Len_i}$までに含まれる事象$\textit{tick}_i$の数)}
\end{align}
としたい。
このとき、表記の方法としては以下の2通りが考えられる。
\begin{align}
&t_i=\sum_{k=1}^{\Len_i}z_e(k),\\
&t_i=\textit{count}_{\pi_i}(0,\Len_i).
\end{align}
%
しかし、DES(TDES)が巡回するようなループ構造をしていた場合、仕様を満たす時間を適切に求めることができない。例えば、ある状態で仕様を満たした後に事象$\textit{tick}_i$を生起させる遷移しかない、という状況が考えられる。

そこで、「仕様を満たし終わった」ことを表す状態を下位TDESの元となっているDESに追加する。
ここでは$i$番目のDES $G_{i,\textit{act}}=(S_{i,\textit{act}},\Sigma_{i,\textit{act}},\delta_{i,\textit{act}},s_{\textit{init},i,\textit{act}},\textit{AP}_{i,\textit{act}},L_{i,\textit{act}})$を考え、更新後のDESを$G_{i,\textit{act}}'=(S_{i,\textit{act}}',\Sigma_{i,\textit{act}}',\delta_{i,\textit{act}}',s_{\textit{init},i,\textit{act}}',\textit{AP}_{i,\textit{act}}',L_{i,\textit{act}}')$と表記する。
具体的なDESの更新を以下に示します。図\ref{c}で示した具体例を一緒に見ながらですと分かりやすいと思います。
%
\begin{itemize}
\item
$S'_{i,\textit{act}}=S_{i,\textit{act}}\cup\{s'\}$。
$s'$は仕様$\phi_i$を全て満たし終わったことを表す。
\item
$\Sigma'_{i,\textit{act}}=\Sigma_{i,\textit{act}}\cup\{e_f,\textit{fin}\}$。
$e_f$と$\textit{fin}$のlower time とupper time は以下のように与える。
\begin{align}
l_{e_f}=0,\ u_{e_f}=\infty,\\
l_{\textit{fin}}=0,\ u_{\textit{fin}}=0.\label{eq:fin}
\end{align}
\req{eq:fin}より、事象\textit{fin}が生起できる状態($s'$)に到達すると、事象$\textit{tick}_i$を生起させることができなくなる。
\item
新たな遷移関数を$\delta'_{i,act}:~S'_{i,\textit{act}}\times \Sigma_{i,\textit{act}}'\to S'_{i,\textit{act}}$とする。これは元の遷移関数$\delta_{i,act}$に以下の遷移則が追加された関数である。
\begin{align}
&\forall s\in S_{i,\textit{act}}, ~\delta'_{i,act}(s,e_f)=s',\\
&\delta'_{i,act}(s',\textit{fin})=s'.\label{delta:fin}
\end{align}
\item
$\textit{AP}_{i,\textit{act}}'=\textit{AP}_{i,\textit{act}}\cup\{\textit{Fin}\}$。
\item
原子命題$\textit{Fin}$は$s'$に割り当てられるとする。
$L_{i,\textit{act}}'$は$L_{i,\textit{act}}$に以下の原子命題と状態の対応関係が追加されたものである。
\begin{align}
L_{i,\textit{act}}'(s')=\textit{Fin}.
\end{align}
%
\end{itemize}
%
\begin{figure}[hbt]
\begin{center}
\subfigure
[もともとのDES $G_{i,act}$。]
{
{\includegraphics[width=6cm]{cTDES_1-crop.pdf}} \label{c1}
}
\hspace{0.3cm}
%
%
\subfigure
[更新したDES。]
{
{\includegraphics[width=6cm]{cTDES_2-crop.pdf}} \label{c2}
}
\hspace{0cm}
%
%
\caption{一つ目の課題に対処するためにDESの構造を更新する。} 
\label{c}
\end{center}
\end{figure}
%
%
\begin{comment}
%
%
更新されたDES $G_{i,act}'$から得られるTDESに与える仕様$\phi_i$に以下の仕様を追加する。
\begin{align}\label{add_spec}
&(\lnot \textit{Fin})U_{[0,\Len_i]}\phi_i,&\Diamond_{[0,\Len_i]}\textit{Fin}.
\end{align}
一つ目の仕様は、もともと与えていた仕様$\phi_i$を満たすまで$s'$に遷移してはいけないことを表す。

したがって、最終的に$i$番目の下位TDESに与える仕様$\phi_i'$は、
\begin{align}
\phi_i'=\phi_i 
\land\left\{(\lnot \textit{Fin})U_{[0,\Len_i]}\phi_i \right\}
\land \Diamond_{[0,\Len_i]}\textit{Fin}
\end{align}
となる。
%
%
\end{comment}
%
こうすることで、仕様を満たし、かつ$\textit{tick}_i$を生起させたくない時に対応できます。
%

\ref{2}では、上位TDESと下位TDESの単位時間の違いを調整している。
%
\section{具体例の案}
前回のミーティングで発案された具体例について定式化します。

上位TDESは平面をグリッド状に区切ったものを表現し、
下位TDESはその位置・場所に対応付けられたタスクを表現すると想定している。
上位TDESの状態(場所)$s_i$には、下位TDESで表現されたタスクが存在していることを表す。

$M'$を$1\leq M'\leq M$を満たす整数とする。
$i_h\in\{1,\ldots, M'\}$について、$s_{i_h}$で行うタスクは必ずどこかのタイミングで完了しなければならないとする。
$i_s\in\{M'+1,\ldots, M\}$について、$s_{i_s}$で行うタスクは可能ならば実行するとする。
そうすると、$G$に与えるHard制約とSoft制約は、パラメーター$m_i\in \mathbb{Z}_{\geq 0},i\in\{1,\ldots,M\}$を用いて以下のように与えられる。
\begin{align}
\phi(\bm{m})=&\bigwedge_{i_h=1}^{M'}\Diamond_{[0,\Len]}\Box_{[0,m_{i_h}]}L(s_{i_h}) \land \phi',\\
\Psi(\bm{m})=&\left\{\Diamond_{[0,\Len]}\Box_{[0,m_{i_s}]}L(s_{i_s})|\ i_s\in\left\{M'+1,\ldots, M\right\}\right\}\cup \Psi'.
\end{align}
ここで、$L$は$G$のラベル関数、$\phi'$はHard制約の一部を表すticked LTL${}_f$式、$\Psi'$はSoft制約の一部とその重みのタプルの集合を表す集合とする。
%$m_i$は、先に求めた$G_i$でHard制約を満たすために必要最低限な時間(tick数)によって決められる。




%\bibliography{TAC_refine} %hoge.bibから拡張子を外した名前
%\bibliographystyle{unsrt} 

\end{document}